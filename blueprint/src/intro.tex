\section{Total Complexes}
\begin{definition}\label{def:total_complex}
    If $A_0^{\bullet}\to\ldots \to A_n^{\bullet}$ is a sequence of maps (with $f_i:A_i\to A_{i+1}$ )of complex such that the composition of two consecutive maps is $0$, then let's denote $[A_0^{\bullet}\to\ldots \to A_n^{\bullet-n}]$ the total complex of this double complex, defined by the following data:\begin{itemize}
        \item The object in degré $k$ is $\bigoplus\limits_{i=0}^n A_i[-i]^k$
        \item The diferential is given by the matrix $\begin{pmatrix}
            d_{A_0} & 0 & \ldots & \ldots & 0\\
            f_0 & -d_{A_1} & \ldots & \ldots & 0\\
            0 & f_1 & \ldots & \ldots & 0\\
            \vdots & \vdots & \ddots &  & \vdots\\
            \vdots & \vdots & & (-1)^{n-1}d_{A_{n-1}} & 0\\
            0 & 0 & \ldots & f_{n-1} & (-1)^nd_{A_n}\\
            \end{pmatrix}$
    \end{itemize}
\end{definition}

\begin{proof}
    One needs to check that the matrix square is $0$. Let $M$ be this matrix and $(i,j)$ be integers.

    \[M^2[i,j]=\sum\limits_{k=1}^n M[i,k]M[k,j]= M[i,i] M[i,j]+M[i,i-1]M[i-1,j]\]

    One can distinguish four cases:\begin{itemize}

        \item If $j$ is not in $\{i-2,i-1,i,\}$ then the two terms are $0$. 
        \item If $j=i$, then $M^2[i,j]=((-1)^i d_{A_i})^2+0=0$. 
        \item If $j=i-1$ then $M^2[i,j]0+(-1)^i d_{A_i}\circ f_i+ (-1)^{i+1}d_{A_{i+1}}\circ f_i=0$ because $f_i$ is a morphism of complex.
        \item If $j=i-2$ then $M^2[i,j]=f_{i}\circ f_{i-1}=0$.
    \end{itemize}
\end{proof}

\begin{remark}
    In particular, if $f:A^{\bullet}\to B^{\bullet}$ is a morphism of complex then $[A^{\bullet}\to B^{\bullet-1}]$ is the cone of the morphism f.
\end{remark}

\begin{lemma}\label{lem:q-iso_iff_cone_acyclic}
    A morphism of complex $f:A^{\bullet}\to B^{\bullet}$ is a quasi isomorphism if and only if, its cone is acyclic.
\end{lemma}

\begin{proof}
    One get's a short exact sequence $0\to B^{\bullet}[-1]\to [A^{\bullet}\to B^{\bullet-1}]\to A^{\bullet} \to 0$
    by using the canonical inclusion and projection over the direct sum. The long exact sequence induced in cohomology is then: \[\ldots H^{k-1}A^{\bullet} \to H^k B^{\bullet}[-1]\to H^k[A^{\bullet}\to B^{\bullet-1}]\to H^k A^{\bullet}\to H^{k+1}B^{\bullet}[-1]\ldots\]
    By using the fact that $H^k B^{\bullet}[-1]=H^{k-1}B^{\bullet}$ one gets:  \[\ldots\to  H^{k-1} A^{\bullet} \to H^{k-1} B^{\bullet}\to H^k[A^{\bullet}\to B^{\bullet-1}]\to H^k A^{\bullet}\to H^kB^{\bullet}\ldots\]

    And then the statement is straightforward by reading the exact sequence.
\end{proof}

\begin{lemma}\label{lem:complex_total_of_three_is_acyclic}
    If a complex $[A^{\bullet}\to B^{\bullet-1}\to C^{\bullet-2}]$ is acyclic then there is a long exact sequence \[\ldots \to H^k A\to H^k B\to H^k C\to \ldots\]
\end{lemma}

\begin{proof}
    \uses{lem:q-iso_iff_cone_acyclic}
    One can see that by construction there is a canonical isomorphism of complexess: $[A^{\bullet}\to B^{\bullet-1}\to C^{\bullet-2}]=[A^{\bullet}\to [B^{\bullet}\to C^{\bullet-1}]^{\bullet-1}]$.

    Then by the previous lemma: $A^{\bullet}\to [B^{\bullet}\to C^{\bullet-1}]$ is a quasi isomorphism. One can then rewrite the long exact sequence in cohomology givent by the short exact sequence $0\to C^{\bullet}[-1]\to [B^{\bullet}\to C^{\bullet-1}]\to B^{\bullet} \to 0$ wich is (as in the previous lemma): \[\ldots\to  H^{k-1} B^{\bullet} \to H^{k-1} C^{\bullet}\to H^k[B^{\bullet}\to C^{\bullet-1}]\to H^k B^{\bullet}\to H^kC^{\bullet}\to \ldots\]. 

    The result is then a long exact sequence : \[\ldots\to  H^{k-1} B^{\bullet} \to H^{k-1} C^{\bullet}\to H^kA^{\bullet}\to H^k B^{\bullet}\to H^k C^{\bullet}\to\ldots\]

\end{proof}